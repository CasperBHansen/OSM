%
% task1.tex
%
% System calls for the Buenos filesystem.
%

\section{Filesystem System Calls}
Userland processes were given access to the console by the system calls
\code{syscall\_read} and \code{syscall\_write} in the first assignment. We
wish to extend this beyond the console, and provide support for reading and
writing to and from the filesystem.

\subsection{Filehandles}
For console input and output standard filehandles were defined as fixed values
\figref{code:std-handles}, but when dealing with the filesystem we must
establish and maintain filehandles dynamically.

\begin{multicols}{2}

    \codefig{std-handles}{proc/syscall.h}{64}{66}
    {Standard filehandle definitions.}
    
    \columnbreak
    
    We do so by associating a process with a list of open files, which were
    opened by that process. This list will be held in the process control
    block, and we be of fixed size. For testing purposes we have given it a
    relatively small size of a maximum of 8.
    
\end{multicols}

This is defined as \code{PROCESS\_MAX\_OPEN\_FILES} on line 54 of
\file{proc/process.h}. The associatation array is populated with filehandles
and their filepaths, a structure which we defined on lines 67--70 of the same
file.

\codefig{revised-pcb}{proc/process.h}{72}{97}
{Code excerpt showing the revised \code{process\_control\_block\_t}.}

We will need to perform maintenance on these arrays as files are opened and
closed. For this we define two accessor functions; one for adding and one for
removing filehandles.

\codefig{process-add-file}{proc/process.c}{353}{364}
{Code excerpt showing \code{process\_add\_open\_file}.}

What we do in \code{process\_add\_open\_file} is that we go through the array
of open files for the current process. For each entry we check for an
available spot. If the \code{file\_handle} is $-1$, then the array index is
vacant and we can place the file entry at that index. We then simply set the
\code{file\_handle} and copy the \code{pathname} string into the entry,
effectively occupying that entry with what was passed to the function. Upon
succesfully occupying an entry we return 0, indicating that no error occurred,
whilst returning 1 should we not find any available entry.

\codefig{process-rem-file}{proc/process.c}{378}{389}
{Code excerpt showing \code{process\_remove\_open\_file}.}

Correspondingly, in \code{process\_remove\_open\_file} we loop through the
array of open files. For each entry of the array we check if the passed handle
is equal to the current entry \code{file\_handle}. If so, then we must vacate
the entry. As the condition we look for in \code{proces\_add\_open\_file} is
whether or not \code{file\_handle} is equal to $-1$, we must set this when
vacating, as well as ensure that the \code{pathname} reference is nullified.

\newpage
Another function we will be needing is one that can query whether or not a
file is open in any process. A possible scenario is that process $p_a$ opens a
file $f$, and another process $p_b$ attempts to delete the file $f$. We reason
that we shouldn't allow $p_b$ to delete the open file $f$, even if it is not
the owner of the file.

\codefig{process-is-file-open}{proc/process.c}{402}{415}
{Code excerpt showing \code{process\_is\_file\_open}.}

In \code{process\_is\_file\_open} we go through the entire array of processes,
and if the process is alive. If that is the case, then we check by the passed
\code{pathname} if the process has an entry for that file. Should the process
have such an entry we return 1 indicating that we found such an entry, if no
such entry is found, we return 0.

A similar, but specialized function, toward the current process is also
provided as we will need this as well --- we are well aware that these could,
with some thought be used as one function.

\codefig{process-is-file-open-in-current-process}{proc/process.c}{430}{439}
{Code excerpt showing \code{process\_is\_file\_open\_in\_current\_process}.}

This function does the exact same as the previously discussed, but only checks
for open files in the current process.

\newpage
\subsection{System Calls}
Now that processes have the necessary components, we can give userland
processes access to the filesystem by implementing the system calls.

\subsubsection{Opening and closing}
To open a file we check that the \code{pathname} is legal on lines 149--153,
and that we could indeed open the file on lines 158--160. If this is the case,
we then attempt to add it to the process open files on lines 162--168 --- note
that we offset the handle by 2, because filehandles in the interval $[0;2]$
are reserved for standard file handles \figref{code:std-handles}. If this
succeeds then we return the handle, otherwise we close the file and return
$-1$ indicating an error.

\codefig{syscall-open}{proc/syscall.c}{147}{169}
{Code excerpt showing the \code{handle\_syscall\_open} implementation.}

\newpage
When closing a file we ensure that we are not attempting to close a standard
filehandle \figref{code:std-handles} on line 114. We then attempt to remove
the file from the process open files. If succesful we can then close the file
on line 118, otherwise, we return $-1$ indicating an error has occurred.

\codefig{syscall-close}{proc/syscall.c}{113}{122}
{Code excerpt showing the \code{handle\_syscall\_close} implementation.}

\subsubsection{Creating and deleting}
For creating files, we did not have to perform any special operations before
delegating the work on to the virtual filesystem API. So, we decided not to
have a function for that, and simply call \code{vfs\_create} directly.

\codefig{syscall-create}{proc/syscall.c}{272}{274}
{Code excerpt showing the \code{handle\_syscall\_create} implementation.}

When deleting a file we make use of the previously mentioned utility function
\figref{code:process-is-file-open}, which checks whether or the file at

\code{pathname} is current open in any process. If it is, return $-1$
indicating an error occurred, otherwise call \code{vfs\_remove} which performs
the actual deletion.

\codefig{syscall-delete}{proc/syscall.c}{132}{134}
{Code excerpt showing the \code{handle\_syscall\_delete} implementation.}

\newpage
\subsubsection{Seeking in files}
Our implementation of seeking may be very inefficient. First we check that the
file is actually open on lines 206-207 and that the offset is legal on lines
209-210. We then reset the seek position, and read until the seek position has
been reached, so as to not seek beyond the file.

\codefig{syscall-seek}{proc/syscall.c}{204}{220}
{Code excerpt showing the \code{handle\_syscall\_seek} implementation.}

\subsubsection{Reading and writing}
As reading and writing from and to standard filehandles
\figref{code:std-handles} was already implementation in a previous assignment,
all we need to do was to handle whenever the filehandle did not correspond to
one of these.

For reading this was a simple matter of checking whether or not the filehandle
was \code{FILEHANDLE\_STDIN} \figref{code:std-handles}, performing the actions
needed for that (which were already in place) and returning from the
\code{if}-statement. If the filehandle was not \code{FILEHANDLE\_STDIN} we
then simply delegate the call on to \code{vfs\_read}.

\codefig{syscall-read}{proc/syscall.c}{182}{190}
{Code excerpt showing the revised \code{handle\_syscall\_read}.}

Similarly, for writing, we handle the cases of \code{FILEHANDLE\_STDOUT} and
\code{FILEHANDLE\_STDERR} in the same manner, requiring the same logical
structure as with read. If these are not the case, we then case any other case
after the \code{if}-statement by delegating the work on to \code{vfs\_write}.

\codefig{syscall-write}{proc/syscall.c}{232}{241}
{Code excerpt showing the revised \code{handle\_syscall\_write}.}

