%
% task1.tex
%
% TLB exception handling in Buenos.
%

\section{TLB Exception Handling}
This section describes the implementation of handling of exceptions caused by
misses in the translation lookaside buffer (TLB). These occur when a page cannot
be found in the TLB, and they are handled by replacing a random entry in the TLB
with the entry that was not found. A more efficient solution would be to
implement a FIFO replacement strategy, rather than a random replacement
strategy, that is, a strategy that replaces the oldest existing entry in
the TLB with the new requested page entry.

\subsection{TLB modified exception}
The TLB modified exception is raised when a memory store operation is requesting
a page whose dirty bit is 0 which means that the page is read-only. If this
occurs in user mode, it is treated as an access violation, an error messsage is
printed and the thread is ended by calling the \verb|thread_finish| function
which frees the memory of the thread and sets its state to dying. If the
thread is running in kernel mode, a kernel panic will occur. The code for the
exception handler is shown in figure \ref{code:tlb_modified_handler} below.

\codefig{tlb_modified_handler}{vm/tlb.c}{74}{86}
{Code excerpt showing TLB modified exception handler function.}

\subsection{TLB load and store exception handling}
A TLB load exception is caused by a memory load operation that requires a page
entry which is not present in the TLB or is marked as invalid. This is handled
by the function \verb|tlb_load_exception|. Likewise, a TLB store exception
occurs when a memory store operation requires a missing page entry or the entry
in the TLB is marked as invalid. This is handled by the function
\verb|tlb_store_exception|. A new function \verb|tlb_load_store| is defined and
called from both of the two mentioned exception handlers, since they are handled
in the same way, and if this function returns a non-zero value, a kernel panic
happens since this means that the page entry could not be found or that is was
invalid. The code for these two exception handler functions is pretty trivial, 
and it is shown in figure \ref{code:tlb_load_handler} and
\ref{code:tlb_store_handler}.

\codefig{tlb_load_handler}{vm/tlb.c}{88}{92}
{Code excerpt showing TLB load exception handler function.}

\codefig{tlb_store_handler}{vm/tlb.c}{94}{98}
{Code excerpt showing TLB store exception handler function.}


\subsection{tlb\_load\_store}
\codefig{tlb_load_store}{vm/tlb.c}{46}{46}
    {Prototype of function behind \code{tlb\_load\_exception} and
                          \code{tlb\_store\_exception}.}

\codefig{badvpn2}{vm/tlb.c}{48}{50}
    {Declaration of state and badvpn2}
A \code{tlb\_exception\_state\_t} is created and the current state is loaded
into it, the pagenumber is also saved to \code{unint32\_t badvpn2}. vpn2
basically means two pages, ie. 8192 bytes or 2 KiB.


\codefig{thread}{vm/tlb.c}{53}{55}
    {Declaration of \code{*t\_table}, \code{*p\_table} and \code{*p\_entries}}
The pointer to the current thread entry is put into \code{*t\_table}, through
that the pointer to the pagetable is found and put into \code{*p\_table}. The
pointer to the pagetables, ie. entries, is then put into \code{*p\_entries}.

\codefig{even_odd}{vm/tlb.c}{57}{57}
    {Declaration of odd and bitwise magic happens.}

\codefig{tlb_ls_for}{vm/tlb.c}{59}{70}
    {A loop through \code{p\_entries} and possibly some declarations.}
A for loop that goes through the entries on the current thread. Then if the
current vpn, found through \code{p\_entries[i]} equals to the \code{badvpn2}
found before and the register returned are bigger than 0, then the current entry's ASID,
\code{p\_entries[i].ASID}, is set to \code{state.asid}, where state is the
exception state. The current entry is then probed and if the result is less than
zero then a call to \code{\_tlb\_write\_random(...)} is issued, otherwise
\code{\_tlb\_write(...)} is called. After this 0 is returned. \\
If the first if-clause is not accepted then \code{tlb\_load\_store\_exception}
returns 1.

\subsection{Other modifications}
As noted in the last comment in \file{kernel/interrupt.c}, the call to the
function \verb|tlb_fill| is not needed after implementing the handling of
exceptions caused by TLB misses. The ASID field in the \verb|tlb_entry_t| struct
must still be set, however, to match the thread id of the current thread. This
is the address space identifier and by setting it to the thread id, it is
ensured that the page entry is valid for the current thread. Thus, the call to
\verb|_tlb_set_asid| in the original \verb|tlb_fill| function, is added to the
end of the \verb|interrupt_handle| function in \file{kernel/interrupt.c},
replacing \verb|tlb_fill|. This is shown in the code excerpt in figure
\ref{code:interrupt_asid}.

\codefig{interrupt_asid}{kernel/interrupt.c}{203}{204}
{Code excerpt showing the call to set ASID in TLB.}

This leaves out the call to \verb|_tlb_write|, which was called from
\verb|tlb_fill|. This call filled the TLB with pages and thereby limiting the
available memory to the size of the TLB. By implementing the exception handling,
the pages are first inserted as needed and newly requested pages replace other
pages in the TLB when they are not available. Thus, the TLB is filled
automatically as needed. Similarly, \verb|tlb_fill| is no longer needed when
creating a new process so the two calls in \file{proc/process.c} are removed.


\subsection{Tests}
\codefig{tlbtests}{tests/tlb_test.c}{1}{36}{TLB test file}

