%
% task2.tex
%
% A simple shell and directory listing support.
%

\section{A Simple Shell}
Up until now the only form of input supported by the system, was to feed the
kernel with at start program and then let it run through everything
automatically. In this task we are given a small shell program and we are
supposed to implement some basic programs, such as ls, rm and cp.

\subsection{Command: exit}
There is not much code behind this command. A \code{syscall\_halt()} is issued
and thus halting the kernel.\\
Since the command requires no input, there is no check for the validness of the
input, since it is simply discarded.
\codefig{exitcommand}{tests/osh.c}{167}{171}{Code excerpt for the exit command.}

\subsection{Command: rm}
The input is checked, if argc is less than two, no input is given and the
operation is halted, since there was not given any input.\\
Then a while-loop is entered, the while loop is in effect while \code{i < argc},
where argc holds the amount of inputs given, ie. \code{rm file1 file2 file3},
means \textit{argc = 4}. The count variable i is set to 1, thus ignoring the
command name itself, then for every increment a file is deleted, if something
went wrong a message is given and the loop keeps on trucking until the exit
criteria is met. 
\codefig{rmcommand}{tests/osh.c}{190}{206}{Code excerpt for the rm command.}

\subsection{Command: cp}
The input is checked, if argc is different than 3 then the operation is halted
and a message on how to correctly use the command is given.\\
If \code{syscall\_open(argv[1])} is less than 3 an error has occured and the
operation is halted. Otherwise a handle of the source file is put into src. Then
it's time to get the size of the source file, this is done in the for-loop where
the criteria is \verb|syscall\_read(src, &dummy, 1)|, meaning that when the end
of file is reached a zero is returned and we get out of the loop again with the
file size in the is kept in the variable \code{size}. The reading position of
\code{src} is set to zero and while loop is initiated and the content is placed
into the buffer.\\
The destination file is created and a check for if the file was created
successfully, if errors have occured then the operation is halted.
\codefig{cpcheck}{tests/osh.c}{140}{145}{Code excerpt from cp command, showing
the destination file check}
The newly created file, codenamed dst, is then opened and if this turns bad then
the source file is closed and the operation halts.\\
Then at last, the stars align, and the data is written to the new file, byte by
byte. If, when done, the \verb|syscall\_write(...)| returns a value different
from \verb|size|, then somethign went bad, the files are closed and the
operation is halted.\\
If the operation is still running, then the copy has been successful and the
files are closed and the operation stops.

\subsection{Command: cmp}
The \code{argc}-variable is checked and if it differs from three, then the
operation is halted and a message is printed showing how to properly use the
command.\\
Then the files are opened, if unsuccessful the operation is halted and the
file(s) are closed. If successful then the for-loop is entered where all the
dirty work happens. If some of the following happen then the files are different
and the loop breaks:\\
\begin{itemize}
  \item \code{buf0 != buf1}\\
        In this case the files are identical right up to byte i, which is
        returned.
  \item \code{read\_ret0 < 0 || read\_ret1 < 0}\\
        An error occured when reading a file, -1 is returned.
  \item \code{read\_ret0 != read\_ret1}
        The file lengths differ ie. they are not alike. -1 is returned.
\end{itemize}
And at last, if everything turned out well thus far the files are indeed
identical and we have arrived at the last byte successfully. The loop is broken
and the files are closed.
\codefig{cmpcommand}{tests/osh.c}{208}{255}{Code excerpt for the cmp command.}

\subsection{Command: ls}


\subsection{Counting files}
This section explains the implementation of a system call, called
\verb|syscall_filecount|, that takes a character pointer as argument and
returns an integer. The syscall returns the number of files in the volume with
the name given as argument, or if it is called with the argument \verb|NULL|,
it returns the number of mounted file system volumes. If the argument character
array is not a mounted volume name or if an error occured, a negative integer
is returned (using the macro error codes defined in \file{vfs.h}), indicating
an error. The counting of file system volume names have not been implemented
yet.

The syscall handling is added to \file{tests/lib.\{h,c\}}. The header file
contains the declaration of the \verb|syscall_filecount| function and
\file{lib.c} contains the implementation, shown in figure
\ref{code:filecount_handling}.

\codefig{filecount_handling}{tests/lib.c}{188}{191}{Filecount syscall handling.}

This function just utilizes the assembly syscall wrapper function
\verb|_syscall| for issuing a syscall instruction from C. This syscall is
handled in \file{proc/syscall.c} and this just passes the parameter onto a new
function \verb|vfs_count|, defined in \file{fs/vfs.c} and declared in the
correspondig header file. This code except from \file{proc/syscall.c} is shown
in figure \ref{code:filecount_syscall}.

\codefig{filecount_syscall}{proc/syscall.c}{279}{281}{Filecount syscall handling.}

The interface, so to speak, to the file system for performing the described
counting task is implemented in the \file{fs/vfs.c} file. That is, the function
\verb|vfs_count|, shown in figure \ref{code:vfs_count}, retrieves the
filesystem of the given volume name and calls the actual \verb|count| function
implmented in the file system that implements the interface specified in the
virtual file system. This design is somewhat object-oriented, allowing for
other file system implementations to be added and used wihtout change, if they
implement the functions specified in, and called from, the virtual file system.

\codefig{vfs_count}{fs/vfs.c}{900}{923}{Filecount virtual file system interface.}

The actual work is performed in the trivial file system, in the new function
\verb|tfs_count|. The trivial file system structure, of type \verb|tfs_t|, is
accessed through the \verb|internal| field in the argument file system
structure. The lock of the file system is acquired, i.e. by calling procure on
the semaphore, securing mutual exclusion on file system access. The
\verb|buffer_md| field is an array of points to type \verb|tfs_direntry_t|,
which contain the inode and name of the file. The array is traversed and for
each entry with an inode that is greater than zero, the return value variable
is incremented. The inode that is greater than zero indicated that the array
entry struct is a valid and existing file in the file system volume.

\codefig{tfs_count}{fs/tfs.c}{822}{836}{Filecount trivial file system implementation.}

After the loop, the semaphore is released and the variable \verb|ret|, that
contains the number of files in the volume, is returned.

\subsection{Retrieving file names}
The sycall called \verb|syscall_file| takes three arguments: a volume name of
type \verb|char| pointer, an integer index and a buffer of type \verb|char|
pointer. As before, \verb|NULL| for the first argument should make the function
work on volume names rather than file names, but this is also not implemented
yet. If the volume name argument is valid, the name of the file at the
specified index in the table of files is copied to the buffer given as third
argument.
