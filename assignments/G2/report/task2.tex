%
% task2.tex
%
% System calls for user-process control in Buenos.
%

\section{System calls}
Here, we will discuss the implementation of user-process control system calls.

\subsection{Executing processes}
When this call is made, we want to spawn a new process with the appropriate
executable. As the executable is passed to this function by way of the user
context, we can simply read out the argument from CPU register {\bf A1}
(\verb|$a1| in \textsc{Mips} jargon) as done on line 2. Then we pass it into a call to
\code{process\_spawn}, the return value of which we need to put back into CPU
register {\bf V0} (\verb|$v0|).

\begin{figure}[H]
    \lstinputlisting{code/handle_syscall_exec.cpp}
    \caption{Code excerpt showing \code{handle\_syscall\_exec}.}
    \label{code:handle_syscall_exec}
\end{figure}


\subsection{Exiting processes}
When a process returns (e.g. \code{return 0}) it will be handled by this
function. The return value of the process lies in the passed user context,
which we retrieve from register {\bf A1} on line 2. We pass this on to the
{\bf V0} register, and then get the process id by which we determine whether
or not to simply finish the process by calling \code{process\_finish} or halt
the system entirely. If the process is the start-up process, then we should
halt, if not simply finish the process.

\begin{figure}[H]
    \lstinputlisting{code/handle_syscall_exit.cpp}
    \caption{Code excerpt showing \code{handle\_syscall\_exit}.}
    \label{code:handle_syscall_exit}
\end{figure}


\subsection{Joining processes}
When this call is made, we want to pass the process id, which lies in CPU
register {\bf A1}, on to the \code{process\_join} function that does the
actual work here. This is done on line 2. The return value of
\code{process\_join} must be put back into CPU register {\bf V0}, which is
done on line 3, along with the actual call.

\begin{figure}[H]
    \lstinputlisting{code/handle_syscall_join.cpp}
    \caption{Code excerpt showing \code{handle\_syscall\_join}.}
    \label{code:handle_syscall_join}
\end{figure}

