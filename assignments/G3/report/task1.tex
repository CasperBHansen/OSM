%
% task1.tex
%
% Task 1: A Thread-Safe Stack
%
\section{Thread-safe multithreaded matrix multiplication}
In this section, the implementation of a thread-safe stack, that can handle
concurrent access by multiple threads in a safe manner, is explained. This is
done using \verb|PThread| mutex locks.

\subsection{A thread-safe stack}
The interface of the stack, as specified
in the task description, is shown in the code excerpt from the header file
\file{stack.h} in figure \ref{code:stack_h}.

\codefig{stack_h}{../assignments/G3/task1/stack.h}{4}{39}
        {Stack header file \file{stack.h}.}

The stack data structure is implemented using a simple singly-linked list,
stored in the \verb|list_t| struct, where \verb|head| is a \verb|void| pointer
to the element stored in the head node of the list and \verb|tail| is a
recursive pointer the data structure, containing the tail of the list.
The stack data structure is implemented with the \verb|stack_t| structure, which
simply contains a pointer to the list structure that stores the elements in the
stack. Hence, the stack can be arbitrarily large and memory is
allocated/deallocated on each push/pop of an element on the stack.

A stack is initialized by setting the struct member \verb|top| to \verb|NULL|,
indicating that the stack is empty, and initializing a \verb|PThread| mutex.
The \verb|stack_init| function is shown in figure \ref{code:stack_init}.

\codefig{stack_init}{../assignments/G3/task1/stack.c}{7}{21}
        {Initialization of a new stack.}

The function call \verb|pthread_mutex_init(&mutex, NULL)| initializes the
globally declared \verb|PThread| mutex with default properties and leaves it
initially unlocked. It is the responsibility of the programmer to allocate
and deallocate memory for the stack data structure, that is, not the
\verb|list_t| structure or the nodes of the list, but just \verb|struct stack_t|.

The function \verb|stack_destroy| serves the purpose of freeing the memory used
for any remaining elements on the stack, by traversing the list and calling
\verb|free| on each \verb|list_t| node, and destroying the mutex lock. To avoid
conflicts with other threads, the mutex is locked by the current thread before
deallocating the elements in the stack. This has the effect, that whenever any
other thread tries to acquire the mutex lock, the call will block and execution
will not continue until the lock is unlocked by the thread that owns it at that
time. After this, the lock is unlocked and destroyed. Depending on the
implementation of the mutex lock, is might not be necessary to unlock the mutex
before destroying it. Also, this current implementation might allow another
thread to lock the mutex and begin work on the, now empty, stack just before the
mutex is destroyed, resulting in undefinded behaviour.

\codefig{stack_destroy}{../assignments/G3/task1/stack.c}{23}{58}
        {Destroying a stack.}

The function \verb|stack_empty| simply checks whether the stack is empty by
checking if the \verb|top| pointer is \verb|NULL|, returning 0 if the stack is
not empty and a positive non-zero integer if the stack is empty. The function
\verb|stack_top| first checks if the stack is empty, returns \verb|NULL| if that
is the case and otherwise returns the pointer given in the \verb|head| member of
the list struct that the \verb|top| member points to. The return type is a
\verb|void| pointer which allows any data to be pointed to and stored in the
list.  The code for these two functions are shown in figure
\ref{code:stack_empty_top}.

%These functions obviously runs in \(O(1)\) since only a constant number of
%comparisons and struct lookups are performed.

\codefig{stack_empty_top}{../assignments/G3/task1/stack.c}{60}{69}
        {\texttt{stack\_empty} and \texttt{stack\_top} functions.}

The functions \verb|stack_pop| and \verb|stack_push| works very similarly. First
the mutex is locked by the currently executing thread by calling
\verb|pthread_mutex_lock(&mutex)|, this locks the mutex and thereby making every
subsequent call block until the thread that locked the mutex unlocks it by
calling \verb|pthread_mutex_unlock(&mutex)|. Between the locking and unlocking,
the actual work on the stack is performed. In \verb|stack_pop|, the head of the
list is returned, the pointer in the stack is updated to the tail pointer in the
head of the list and the memory of the head element is freed. This
function is shown in the following code excerpt in figure \ref{code:stack_pop}.

\codefig{stack_pop}{../assignments/G3/task1/stack.c}{71}{99}
{Function \texttt{stack\_pop}. Pop top element of stack.}

If the stack is empty, the call to \verb|stack_pop| will return \verb|NULL| and
this will be done without ever locking a mutex since the call to
\verb|stack_emtpy| does not lock a mutex either and the function returns if that
is the case.

The function \verb|stack_push| also locks and unlocks the mutex lock before and
after the actual operation on the stack. Memory is allocated for a new list
element using \verb|malloc|, the head pointer in this new element is set to
point to the element being inserted. The tail pointer in the new element is
assigned the address of the element on the top of the stack and the \verb|top|
pointer in the stack struct is assigned the address of the new list element.
This is shown in figure \ref{code:stack_push}.

\codefig{stack_push}{../assignments/G3/task1/stack.c}{101}{125}
{Function \texttt{stack\_push}. Push a new element onto the stack.}


\subsection{Multi threaded matrix multiplication}
We have implemented a matrix multiplication function that is easy to use in a
multi threaded environment and reap its benefits, ie. starting several threads
and then combining the results, also known as parallel programming.\\
\\
\codefig{matrix_t-struct}{../assignments/G3/task1/matmult.c}{10}{13}{C struct for
a matrix}
where the integer pointer \textsf{mat} points to the first number in the matrix,
\textsf{m} is its rows and \textsf{n} is its columns and are integers.\\
The \textsf{struct task\_t} is a task, which is sent to the thread/processor:\\
\codefig{task_t-struct}{../assignments/G3/task1/matmult.c}{15}{18}{C struct for
a task}
The two fields \textsf{lhs} and \textsf{rhs} are matrix\_t pointers, which holds the two
matrices that are to be multiplicated and the integers \textsf{i} and \textsf{j} hold
the current position.\\
\\
The multiplication function:
\codefig{matmult}{../assignments/G3/task1/matmult.c}{43}{60}{Multiplication
function for matrices}
The function takes a void pointer, this is the task, it's then cast to a task\_t
pointer. If the first matrix's, lhs, row number and the second matrix's, rhs,
column number do not match, the function exits since, it's not a valid matrix
multiplication.\\
Then the row of LHS-matrix and the column of RHS-matrix are multiplied and
put into the integer \textsf{ret} after this \textsf{pthread\_exit((void *)
ret);} is issued, returning the value through a void pointer.\\

\codefig{main_decl}{../assignments/G3/task1/matmult.c}{62}{77}{main function,
section where declarations happen}
Nothing interesting happens here, some declarations and the two matrices that
are to be multiplied are printed.\\

\codefig{main_stack}{../assignments/G3/task1/matmult.c}{80}{90}{main function,
section where row-column multiplication tasks are pushed onto the stack.}
The first thing inside the second for loop to happen is the creation of a task.
One task for each lhs.row or rhs.column, depending on your philosophical bend.
The matrices a and b, defined earlier, are placed into the newly created task,
then the integers \textsf{x} and \textsf{y}, from the loop counter, are placed
into the task, \textsf{task->i = x} and \textsf{task->j = y}. And at last the
task is ready to be pushed onto the stack.\\

\codefig{main_work_init}{../assignments/G3/task1/matmult.c}{92}{102}{main function, 
variable initialization and some memory allocation.}
Nothing interesting here apart from the initialization of the matrix that is
destined to be filled with results from the matrix multiplication.

\codefig{main_work}{../assignments/G3/task1/matmult.c}{104}{117}{main function, 
thread handling.}
While the stack, holding the aforementioned tasks, is not empty then the integer
i is set to zero and three loops are to be gone through.\\
The first loop, a while loop, pops a certain number of tasks of the stack and
places them into an array and exists.\\
The next loop, a for loop, creates the threads and assigns the work which was
placed onto the array \textsf{tasks} and the function to be carried out on said
tasks is also defined, in this case \textsf{row\_col\_mult} which is defined
above.\\
The last loop fetches the result from the threads with the function
\textsf{pthread\_join} and puts the result into the matrix \textsf{res\_mat}.
After that the array tasks is deallocated.\\

\codefig{main_work}{../assignments/G3/task1/matmult.c}{119}{127}{main function, 
prints the resulting matrix and frees the stack.}
This last bit of code prints the resulting matrix and deallocates the stack.
