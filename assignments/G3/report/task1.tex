%
% task1.tex
%
% ...
%

\section{Task 1}
...


\subsection{Multi threaded matrix multiplication}
We have implemented a matrix multiplication function that is easy to use in a
multi threaded environment and reap its benefits, ie. starting several threads
and then combining the results, also known as parallel programming\\
\\.
The \textsf{struct matrix\_t} is defined as follows:\\
\codefig{matrix_t-struct}{../assignments/G3/task1/matmult.c}{10}{13}{C struct for
a matrix}
where the integer pointer \textsf{mat} points to the first number in the matrix,
\textsf{m} is its rows and \textsf{n} is its columns and are integers.\\
The \textsf{struct task\_t} is a task, which is sent to the thread/processor:\\
\codefig{task_t-struct}{../assignments/G3/task1/matmult.c}{15}{18}{C struct for
a task}
The two fields \textsf{lhs} and \textsf{rhs} are matrix\_t pointers, which holds the two
matrices that are to be multiplicated and the integers \textsf{i} and \textsf{j} hold
the current position.\\
\\
The multiplication function:
\codefig{matmult}{../assignments/G3/task1/matmult.c}{43}{60}{Multiplication
function for matrices}
The function takes a void pointer, this is the task, it's then cast to a task\_t
pointer. If the first matrix's, lhs, row number and the second matrix's, rhs,
column number do not match, the function exits since, it's not a valid matrix
multiplication.


