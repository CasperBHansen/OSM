%
% task1.tex
%
% Task 1: A Thread-Safe Stack
%
\section{Thread-safe multithreaded matrix multiplication}
In this section, the implementation of a thread-safe stack, that can handle
concurrent access by multiple threads in a safe manner, is explained. This is
done using \verb|PThread| mutex locks.

\subsection{A thread-safe stack}
The interface of the stack, as specified
in the task description, is shown in the code excerpt from the header file
\file{stack.h} in figure \ref{code:stack_h}.

\codefig{stack_h}{../assignments/G3/task1/stack.h}{4}{39}
        {Stack header file \file{stack.h}.}

The stack data structure is implemented using a simple singly-linked list,
stored in the \verb|list_t| struct, where \verb|head| is a \verb|void| pointer
to the element stored in the head node of the list and \verb|tail| is a
recursive pointer the data structure, containing the tail of the list.
The stack data structure is implemented with the \verb|stack_t| structure, which
simply contains a pointer to the list structure that stores the elements in the
stack. Hence, the stack can be arbitrarily large and memory is
allocated/deallocated on each push/pop of an element on the stack.

A stack is initialized by setting the struct member \verb|top| to \verb|NULL|,
indicating that the stack is empty, and initializing a \verb|PThread| mutex.
The \verb|stack_init| function is shown in figure \ref{code:stack_init}.

\codefig{stack_init}{../assignments/G3/task1/stack.c}{7}{21}
        {Initialization of a new stack.}

The function call \verb|pthread_mutex_init(&mutex, NULL)| initializes the
globally declared \verb|PThread| mutex with default properties and leaves it
initially unlocked. It is the responsibility of the programmer to allocate
and deallocate memory for the stack data structure, that is, not the
\verb|list_t| structure or the nodes of the list, but just \verb|struct stack_t|.

The function \verb|stack_destroy| serves the purpose of freeing the memory used
for any remaining elements on the stack and destroying the mutex lock. To avoid
conflicts with other threads, the mutex is locked by the current thread before
deallocating the elements in the stack. This has the effect, that whenever any
other thread tries to acquire the mutex lock, the call will block and execution
will not continue until the lock is unlocked by the thread that owns it at that
time. After this, the lock is unlocked and destroyed. Depending on the
implementation of the mutex lock, is might not be necessary to unlock the mutex
before destroying it. Also, this current implementation might allow another
thread to lock the mutex and begin work on the, now empty, stack just before the
mutex is destroyed, resulting in undefinded behaviour.

\codefig{stack_init}{../assignments/G3/task1/stack.c}{23}{58}
        {Destroying a stack.}

\subsection{Multi threaded matrix multiplication}
We have implemented a matrix multiplication function that is easy to use in a
multi threaded environment and reap its benefits, ie. starting several threads
and then combining the results, also known as parallel programming.\\
\\
\codefig{matrix_t-struct}{../assignments/G3/task1/matmult.c}{10}{13}{C struct for
a matrix}
where the integer pointer \textsf{mat} points to the first number in the matrix,
\textsf{m} is its rows and \textsf{n} is its columns and are integers.\\
The \textsf{struct task\_t} is a task, which is sent to the thread/processor:\\
\codefig{task_t-struct}{../assignments/G3/task1/matmult.c}{15}{18}{C struct for
a task}
The two fields \textsf{lhs} and \textsf{rhs} are matrix\_t pointers, which holds the two
matrices that are to be multiplicated and the integers \textsf{i} and \textsf{j} hold
the current position.\\
\\
The multiplication function:
\codefig{matmult}{../assignments/G3/task1/matmult.c}{43}{60}{Multiplication
function for matrices}
The function takes a void pointer, this is the task, it's then cast to a task\_t
pointer. If the first matrix's, lhs, row number and the second matrix's, rhs,
column number do not match, the function exits since, it's not a valid matrix
multiplication.\\
Then the row of LHS-matrix and the column of RHS-matrix are multiplied and
put into the integer \textsf{ret} after this \textsf{pthread\_exit((void *)
ret);} is issued, returning the value through a void pointer.\\

\codefig{main_decl}{../assignments/G3/task1/matmult.c}{62}{77}{main function,
section where declarations happen}
Nothing interesting happens here, some declarations and the two matrices that
are to be multiplied are printed.\\

\codefig{main_stack}{../assignments/G3/task1/matmult.c}{80}{90}{main function,
section where row-column multiplication tasks are pushed onto the stack.}
The first thing inside the second for loop to happen is the creation of a task.
One task for each lhs.row or rhs.column, depending on your philosophical bend.
The matrices a and b, defined earlier, are placed into the newly created task,
then the integers \textsf{x} and \textsf{y}, from the loop counter, are placed
into the task, \textsf{task->i = x} and \textsf{task->j = y}. And at last the
task is ready to be pushed onto the stack.\\

\codefig{main_work_init}{../assignments/G3/task1/matmult.c}{92}{102}{main function, 
variable initialization and some memory allocation.}
Nothing interesting here apart from the initialization of the matrix that is
destined to be filled with results from the matrix multiplication.

\codefig{main_work}{../assignments/G3/task1/matmult.c}{104}{117}{main function, 
thread handling.}
While the stack, holding the aforementioned tasks, is not empty then the integer
i is set to zero and three loops are to be gone through.\\
The first loop, a while loop, pops a certain number of tasks of the stack and
places them into an array and exists.\\
The next loop, a for loop, creates the threads and assigns the work which was
placed onto the array \textsf{tasks} and the function to be carried out on said
tasks is also defined, in this case \textsf{row\_col\_mult} which is defined
above.\\
The last loop fetches the result from the threads with the function
\textsf{pthread\_join} and puts the result into the matrix \textsf{res\_mat}.
After that the array tasks is deallocated.\\

\codefig{main_work}{../assignments/G3/task1/matmult.c}{119}{127}{main function, 
prints the resulting matrix and frees the stack.}
This last bit of code prints the resulting matrix and deallocates the stack.
