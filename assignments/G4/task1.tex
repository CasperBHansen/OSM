%
% task1.tex
%
% TLB exception handling in Buenos.
%

\section{TLB Exception Handling}
This section describes the implementation of handling of exceptions caused by
misses in the translation lookaside buffer (TLB). These occur when a page cannot
be found in the TLB, and they are handled by replacing a random entry in the TLB
with the entry that was not found. A more efficient solution would be to
implement a FIFO replacement strategy, rather than a random replacement
strategy, that is, a strategy that replaces the oldest existing entry in
the TLB with the new requested page entry.


\subsection{TLB modified exception}
The TLB modified exception is raised when a memory store operation is requesting
a page whose dirty bit is 0 which means that the page is read-only. If this
occurs in user mode, it is treated as an access violation, an error messsage is
printed and the thread is ended by calling the \verb|thread_finish| function
which frees the memory of the thread and sets its state to dying. If the
thread is running in kernel mode, a kernel panic will occur. The code for the
exception handler is shown in listing \ref{code:tlb_modified_handler} below.

\codefig{tlb_modified_handler}{vm/tlb.c}{74}{86}
{Code excerpt showing TLB modiefied exception handler function.}


\subsection{Tests}
...
