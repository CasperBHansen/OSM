\documentclass[11pt]{article}

\usepackage{a4wide}             % save a few rain forests
\usepackage{amsmath,amssymb}    % we can do maths 'n shit
\usepackage{color}              % so purrty
\usepackage{float}              % put things exactly where I tell you!
\usepackage[utf8]{inputenc}     % i can has UTF-8
\usepackage{listings}           % pretty-colored code excerpts
\usepackage{multicol}           % such column.. very multi.. wow!

\definecolor{comment}{rgb}      {0.38, 0.62, 0.38}
\definecolor{keyword}{rgb}      {0.10, 0.10, 0.81}
\definecolor{identifier}{rgb}   {0.00, 0.00, 0.00}
\definecolor{string}{rgb}       {0.50, 0.50, 0.50}

\newcommand{\code}[1]{{\tt #1}}
\newcommand{\file}[1]{{\tt #1}}
\newcommand{\imp}{\rightarrow}

\lstset
{
    language=C,
    % general settings
    numbers=left,
    frame=single,
    basicstyle=\footnotesize\ttfamily,
    tabsize=2,
    breaklines=true,
    % syntax highlighting
    commentstyle=\color{comment},
    keywordstyle=\color{keyword},
    identifierstyle=\color{identifier},
    stringstyle=\color{string},
}

\title
{
    {\Large Group Assignment 1} \\
    Operating Systems and Multiprogramming
}

\author
{
    Anders Kiel Hovgaard \\
    University of Copenhagen \\
    Department of Computer Science \\
    {\tt hzs554@alumni.ku.dk}
    \and
    Casper B. Hansen \\
    University of Copenhagen \\
    Department of Computer Science \\
    {\tt fvx507@alumni.ku.dk}
    \and
    Rúni Klein Hansen \\
    University of Copenhagen \\
    Department of Computer Science \\
    {\tt cdn768@alumni.ku.dk}
}

\date{Last revision: \today}

\begin{document}

\clearpage
\maketitle
\thispagestyle{empty}
\begin{multicols}{2}
    \begin{abstract}
    ...
    \end{abstract}
    \vfill\columnbreak
    \tableofcontents\vfill
\end{multicols}
\newpage

\section{Space-efficient doubly-linked list}
We will begin our discussion of our implementation of the space-efficient
doubly-linked list by making a few remarks. Note that because of the
implementations extensive use of the XOR-operation, and the fact that this
operation requires type-casting in all occurences of its use, we have declared
a preprocessor definition macro \code{XOR\_PTR} that relieves much repetitive
typing and makes the code a lot more readable -- an excerpt of the code
showing this macro has been provided below.

\begin{lstlisting}
#define XOR_PTR(a, b) ((node *) ((uintptr_t) (a) ^ (uintptr_t) (b)))
\end{lstlisting}

It should be noted that uses of preprocessor definitions are expanded by the
preprocessor before compilation of the code occurs.\\
\\
The data structure to hold the list is given in the following code excerpt. The
type \code{item} is defined as a synonym for \code{void}, meaning that a
\code{item} pointer will hold any type of element.
\begin{lstlisting}
typedef struct node {
    item * thing;
    struct node * ptr;
} node;

typedef struct dlist {
    node * head, * tail;
} dlist;
\end{lstlisting}
The struct, \code{dlist}, contains a pointer, of type \code{node} pointer, to
the head and one to the tail of the list. Another struct of type \code{node}
contains a \code{item} (\code{void}) pointer to the \code{item} in the node and
a recursive \code{node} pointer. This \code{node} pointer contains a pointer
that is computed by applying the XOR operator (\lstinline{^} as applied to
pointer variables using the aforementioned \lstinline{XOR_PTR} macro) to the
node before and the node after the node. E.g., given a list of nodes
\verb|A B C|, this allows the pointer to node C to be calculated from the node
pointer \code{ptr} in \verb|B| and the address of \verb|A| as the pointer
\code{A XOR ptr}. The head and tail of a list contains the address of the second
first or second last node, respectively, since the node before the head or after
the tail, conceptually speaking, is just \code{NULL} and \code{NULL} is an
identity element of the XOR operator, i.e. \code{NULL XOR X = X} for any
\code{X}.

\subsection{Insertion}
The insertion operation is implemented as the following function declaration, as
specified in the task description.

\begin{lstlisting}
void insert(dlist *this, item *thing, bool atTail);
\end{lstlisting}

The function takes as arguments, a pointer to a doubly-linked list, a pointer to
the item to be inserted in the list and a boolean (synonym for \code{int}) value
indicating whether to insert at the head or at the tail of the list.
Memory is allocated for the new node using \code{malloc} and the structure
member \code{thing} is set to the the item given as argument, which places a
pointer to the list element in the new node. If the list is empty, i.e. if the
head pointer is \code{NULL}, the head and tail pointers in the list are updated
to the address of the inserted node and the \verb|XOR|ed pointer in the node is set to
\code{NULL} since there are no nodes before or after it. Otherwise, that is if
the list contains at least one node, the pointer in the new node is assigned the
value of head/tail pointer. The pointer in the current head/tail node is updated
to contain a pointer to the new node XORed the pointer stored in the current
head/tail, since that already contains the address of the next/previous node.
The head/tail pointer in the list is updated to the address of the inserted
node. The \code{insert} function is shown below.

\begin{lstlisting}
void insert(dlist *this, item *thing, bool atTail) {
    // initialize new node
    node *new = malloc(sizeof(node));
    new->thing = thing;

    // empty list
    if (!this->head) {
        new->ptr = NULL;
        this->head = this->tail = new;
    }
    // if we're appending to the tail of the list
    else if (atTail) {
        // XORed pointer is just this->tail since NULL XOR tail = tail
        new->ptr = this->tail;
        // this->tail->ptr contains the address of the node before tail
        this->tail->ptr = XOR_PTR(new, this->tail->ptr);
        this->tail = new;
    }
    // if we're appending to the head of the list
    else {
        // XORed pointer is just this->head since NULL XOR head = head
        new->ptr = this->head;
        // this->head->ptr contains the address of the node after head
        this->head->ptr = XOR_PTR(new, this->head->ptr);
        this->head = new;
    }
}
\end{lstlisting}

Since the code only contains conditional expressions and a fixed number of
assignments, it is clear that its running time is \(O(1)\), i.e. constant, which
is exactly what was required by our boss at Jyrki Katajainen and Company.

\subsection{Extraction}
The extraction implementation is only dependant on the head and the tail. Thus
there is no chain of XORed pointers that have to be updated, only the immediate 
next node has to be updated and the dlist, ie. the struct pointing to the head
and tail. Thus the time complexity is fixed, ie. constant time.\\
Firsts checks if the list is empty, if so, return NULL. The next block is
triggered if the atTail is true, ie. an extraction of the tail of the list. The
next block is mostly the same, but extracts the head, so one explanation will
suffice:
The tail or head is assigned to a temporary variable n, and then ret is assigned
the value helt by tail or head. A new node next is assigned the updated XORed
pointer, if the next is empty, ie. the list is empty, the tail is set to NULL. 
And then the next element in the list is updated to point to the right node.
The node, if necessary, is deallocated.
\begin{lstlisting}
item * extract(dlist * this, bool atTail) {
    item * ret = NULL;
    if (!this->head) {
        return NULL;
    }
    else if (atTail) {
        // retrieve the item
        node * n = this->tail;
        ret = n->thing;
        // correct the head and tail
        node * prev = (node *) XOR_PTR(n->ptr, NULL);
        if (!prev) this->head = NULL;
        else prev->ptr = (node *) XOR_PTR(n, XOR_PTR(prev->ptr, NULL));
        this->tail = prev;
        // deallocate the node
        if (n) free(n);
        n = NULL;
    }
    else {
        // retrieve the item
        node * n = this->head;
        ret = n->thing;
        // correct the head and tail
        node * next = (node *)XOR_PTR(n->ptr, NULL);
        if (!next) this->tail = NULL;
        else next->ptr = (node *) XOR_PTR(n, XOR_PTR(next->ptr, NULL));
        this->head = next;
        // deallocate the node
        if (n) free(n);
        n = NULL;
    }
    return ret;
}
\end{lstlisting}

\subsection{Reverse}
There is only ever going to be three assignments, thus it will always take a
fixed amount of time, ie. $O(3) = O(1)$.\\
What is happening is that the node head$_{this}$ is assigned to a temporary
variable tmp, tail$_{this}$ is then assigned to head$_{this}$ and at last the
tail$_{this}$ is assigned the value of tmp.

\begin{lstlisting}
void reverse(dlist * this) {
    node * tmp = this->head;
    this->head = this->tail;
    this->tail = tmp;
}
\end{lstlisting}

\subsection{Search}
...

\newpage
\section{Buenos}
For the implementation of read/write support in the Buenos operating system,
we chose to keep the extensions within the \file{proc/syscall.c}. We declared
functions in \file{proc/syscall.h} that have to do specifically with the
system call being handled. The following code excerpt are the declarations
found in \file{proc/syscall.h}.

\begin{lstlisting}
void handle_syscall_read(context_t * user_context);
void handle_syscall_write(context_t * user_context);
\end{lstlisting}

Preliminary preparations were done by adding cases for the two new system
calls, namely \code{SYSCALL\_READ} and \code{SYSCALL\_WRITE}, in the switch
statement. In these we simply called the newly declared functions, whose
bodies were left empty. Additionally, after running a simple test checking
whether or not these were actually called, we discovered that the program was
not being shutdown gracefully as we were getting a kernel panic of an
unhandled system call case, which after printing the case value we found to be
the \code{SYSCALL\_EXIT} case. We decided to add it to the switch statement
and handle it by simply replicating the body of \code{SYSCALL\_HALT}, although
we recognize that it may require additional implementation.

\subsection{Read/write commonalities}
Since the read and write functions have much in common, we will address these
common elements, like initialization, before moving on to their respective
implementations in which they differ.

Both functions require us to extract the arguments stored in CPU registers
\code{A1}, \code{A2} and \code{A3}, and since both functions use the arguments
for the same basic purpose, the types remain the same and hence initialization
of both functions also remains the same. An excerpt of the initialization code
of both \code{handle\_syscall\_read} and \code{handle\_syscall\_write} is
provided below.

\begin{lstlisting}
uint32_t fhandle = user_context->cpu_regs[MIPS_REGISTER_A1];
uint8_t * buffer = user_context->cpu_regs[MIPS_REGISTER_A2];
int length = user_context->cpu_regs[MIPS_REGISTER_A3];
\end{lstlisting}

The overall structure of both functions are more or less the same; check
whether or not we are reading or writing from a source or destination that is
supported (or required by the assignment). Otherwise we write an error code to
register \code{V0}. Furthermore, in both cases we need to get the generic
character device, as done on lines 2--3, which is almost shamelessly stolen
from the \code{init\_startup\_fallback} function found in \file{init/main.c}.
In either case, we have to return the length either read or written, which is
done on line 5 --- note that the \code{len} variable isn't declared yet.

\begin{lstlisting}
if (fhandle == /* supported source or destination */) {
    device_t * dev = device_get(YAMS_TYPECODE_TTY, 0);
    gcd_t * gcd = (gcd_t *) dev->generic_device;
    // system call specific code
    user_context->cpu_regs[MIPS_REGISTER_V0] = len;
}
else {
    user_context->cpu_regs[MIPS_REGISTER_V0] = -1;
}
\end{lstlisting}

That concludes common elements of the read and write function implementations.

\subsection{Implementing read support}
In the case of read, we would like to ensure that the source, which is held
in the \code{fhandle}, is equal to \code{FILEHANDLE\_STDIN}. Knowing that the
source is correct, we can add the actual code specific to the read system
call.

Substituting the comment, we simply call the generic character device function
pointer for reading, using the parameters passed. The final read function
implementation is then as shown below.

\begin{lstlisting}
void handle_syscall_read(context_t * user_context) {
    uint32_t fhandle = user_context->cpu_regs[MIPS_REGISTER_A1];
    uint8_t * buffer = user_context->cpu_regs[MIPS_REGISTER_A2];
    int length = user_context->cpu_regs[MIPS_REGISTER_A3];

    if (fhandle == FILEHANDLE_STDIN) {
        device_t * dev = device_get(YAMS_TYPECODE_TTY, 0);
        gcd_t * gcd = (gcd_t *) dev->generic_device;
        int len = gcd->read(gcd, buffer, length);
        user_context->cpu_regs[MIPS_REGISTER_V0] = len;
    }
    else {
        user_context->cpu_regs[MIPS_REGISTER_V0] = -1;
    }
}
\end{lstlisting}

\subsection{Implementing write support}
For the write case, we wish to ensure that the destination, which is held in
the \code{fhandle}, is equal to \code{FILEHANDLE\_STDOUT}. In much the same
way as with read, we can now safely assume that we have a valid destination
and substituting the comment, we simply call the generic character device
function pointer for writing, we get the final write function implementation.

\begin{lstlisting}
void handle_syscall_read(context_t * user_context) {
    uint32_t fhandle = user_context->cpu_regs[MIPS_REGISTER_A1];
    uint8_t * buffer = user_context->cpu_regs[MIPS_REGISTER_A2];
    int length = user_context->cpu_regs[MIPS_REGISTER_A3];

    if (fhandle == FILEHANDLE_STDOUT) {
        device_t * dev = device_get(YAMS_TYPECODE_TTY, 0);
        gcd_t * gcd = (gcd_t *) dev->generic_device;
        int len = gcd->write(gcd, buffer, length);
        user_context->cpu_regs[MIPS_REGISTER_V0] = len;
    }
    else {
        user_context->cpu_regs[MIPS_REGISTER_V0] = -1;
    }
}
\end{lstlisting}

\end{document}

