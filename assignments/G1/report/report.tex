\documentclass[11pt]{article}

\usepackage{a4wide}             % save a few rain forests
\usepackage{amsmath,amssymb}    % we can do maths 'n shit
\usepackage{color}              % so purrty
\usepackage{float}              % put things exactly where I tell you!
\usepackage[utf8]{inputenc}     % i can has UTF-8
\usepackage{listings}           % pretty-colored code excerpts
\usepackage{multicol}           % such column.. very multi.. wow!

\definecolor{comment}{rgb}		{0.38, 0.62, 0.38}
\definecolor{keyword}{rgb}		{0.10, 0.10, 0.81}
\definecolor{identifier}{rgb}	{0.00, 0.00, 0.00}
\definecolor{string}{rgb}		{0.50, 0.50, 0.50}

\newcommand{\code}[1]{{\tt #1}}
\newcommand{\file}[1]{{\tt #1}}
\newcommand{\imp}{\rightarrow}

\lstset
{
    language=C++,
	% general settings
	numbers=left,
	frame=single,
	basicstyle=\footnotesize\ttfamily,
	tabsize=2,
	breaklines=true,
	% syntax highlighting
	commentstyle=\color{comment},
	keywordstyle=\color{keyword},
	identifierstyle=\color{identifier},
	stringstyle=\color{string},
}

\title
{
    {\Large Group Assignment 1} \\
    Operating Systems and Multiprogramming
}

\author
{
    Anders Kiel Hovgaard \\
    University of Copenhagen \\
    Department of Computer Science \\
    {\tt abs123@alumni.ku.dk}
    \and
    Casper B. Hansen \\
    University of Copenhagen \\
    Department of Computer Science \\
    {\tt fvx507@alumni.ku.dk}
    \and
    Rúni Klein Hansen \\
    University of Copenhagen \\
    Department of Computer Science \\
    {\tt abc123@alumni.ku.dk}
}

\date{last revision \today}

\begin{document}

\clearpage
\maketitle
\thispagestyle{empty}
\begin{multicols}{2}
    \begin{abstract}
    ...
    \end{abstract}
    \vfill\columnbreak
    \tableofcontents\vfill
\end{multicols}
\newpage

\section{Space-efficient doubly-linked list}
We will begin our discussion of our implementation of the space-efficient
doubly-linked list by making a few remarks. Note that because of the
implementations extensive use of the XOR-operation, and the fact that this
operation requires type-casting in all occurences of its use, we have declared
a preprocessor definition macro \code{XOR\_PTR} that relieves much repetitive
typing and makes the code a lot more readable--- an excerpt of the code
showing this macro has been provided below.

\begin{lstlisting}
#define XOR_PTR(a, b) ( (uintptr_t)a ^ (uintptr_t)b )
\end{lstlisting}

It should be noted that preprocessor definitions are expanded by the
preprocessor before complication of the code occurs.

\subsection{Insertion}
...

\subsection{Extraction}
...

\subsection{Reverse}
...

\subsection{Search}
...

\section{Buenos}
...

\subsection{Implementing read support}
...

\subsection{Implementing write support}
...

\end{document}

