%
% task1.tex
%
% Types and functions for userland processes in Buenos.
%

\section{Types and functions}
Here, we will discuss the implementation of userland processes in Buenos, by
first examining the code which went into describing processes and how we
manange them in the kernel.

\subsection{Types}
Before we can implement any functions we must declare a data structure with
which we can keep track of individual processes properties, also called a
{\it process control block}. Ours is declared in the code excerpt below.

We will describe the internals of it in short here, but their existence is
reasoned elsewhere, as it makes more sense to justify each one when they are
required throughout the implementation.

\begin{figure}[H]
    \lstinputlisting{code/process_control_block_t.cpp}
    \label{code:process_control_block_t}
    \caption{Code excerpt showing \code{process\_control\_block\_t}.}
\end{figure}

Inside the \code{process\_control\_block\_t}, we have a string which will hold
the associated executable on line 2, the length of which is fixed, such that
process names cannot exceed this upper bound. On line 3 we have state
variable, which is discussed in detail under
{\ref{sec:types|sub:process-states}. Processes can spawn sub-processes, we
argue that such relationships require keeping track thereof, which is the
\code{parent\_id}'s intended use. Processes are associated with a thread, each
process should keep the associated thread id, and so it is with
\code{thread\_id}. The \code{retval} variable will contain the exit-code of a
process. That is, the value returned by the program itself (e.g. \code{return
0}).

Note that the \code{usage} structure remains unused in our implementation, but
is included for later implementation. Its purpose is to keep track of the time
spend executing the process.

\subsection{Process states}
\label{sec:types|sub:process-states}
Throughout the lifecycle of a process it can take on one of several states.
At the time of creation \figref{code:process_create_process} its state is set
as being {\it new}, meaning that the process has simply been created. Once the
process is started \figref{code:process_start} its state will is changed to
reflect this by setting it to be {\it running}. Once the process finishes its
state is set as being {\it dead}, which allows its process id to be reused
\figref{code:process_get_available_pid}.
\begin{figure}[H]
    \lstinputlisting{code/process_state_t.cpp}
    \label{code:process_state_t}
    \caption{Code excerpt showing \code{process\_state\_t}.}
\end{figure}
The remainder of the states are used while the process is still alive. Should
the process need to {\it wait} for something to happen, for instance if it
\code{syscall\_join} \figref{code:process_join}, then it is set as such.

\newpage
\subsection{Functions}
Now that we have our data structures in place, we can put them to use by
discussing the necessary functions that we will be using to manage our
processes.

\subsubsection{Initializing processes}
We chose a rather simple way to initialize the userland process system;
firstly, we reset the associated spinlock (\code{process\_table\_slock}) on
line 2. Inside the spinlock acquisition on line 4 and the following release
of it on line 10, we loop through the process table, setting each process to
a state of being dead.

\begin{figure}[H]
    \lstinputlisting{code/process_init.cpp}
    \label{code:process_init}
    \caption{Code excerpt showing \code{process\_init}.}
\end{figure}

Having done this, we have a condition for which available process id's can be
retrieved. That is, once dead a process can be relieved of its id
\figref{code:process_get_available_pid}, which is discussed later.

\subsubsection{Spawning processes}
The procedure required to spawn a process involves; {\it creating} a process
\figref{code:process_create_process} as can be seen on line 6, creating a
{\it thread} with which to {\it run} the process. In our case, because of time
constraints, we ended up using what was intended as a temporary solution by
which the initial program can be treated as a process as well. The way we went
about doing this is highly inefficient, as we check to see if the process id
created is equal to the start-up process id (which is asserted to be 0,
reasoned by the fact that it is the first process to be initiated). If this is
the case, then we wish to continue executing the process from the current
thread, by calling \code{process\_start} directly. If this is not the case,
then the process must be run from a separate thread, which is created on line
19, and associated with the process on the following line. Running the thread
is delegated on to the threading system by calling \code{thread\_run}.
\begin{figure}[H]
    \lstinputlisting{code/process_spawn.cpp}
    \caption{Code excerpt showing \code{process\_spawn}.}
    \label{code:process_spawn}
\end{figure}

\subsubsection{Creating processes}
We decided to encapsulate the creation of processes as a function of its own,
for the simple reason of readability.
\begin{figure}[H]
    \lstinputlisting{code/process_create_process.cpp}
    \caption{Code excerpt showing \code{process\_create\_process}.}
    \label{code:process_create_process}
\end{figure}

\subsubsection{Acquiring available PID's}
To acquire an available process id, in our implementation, we need to look
through the process table. If we encounter a process which is dead, then that
process no longer needs any of its data and can be relieved of its process id.
\begin{figure}[H]
    \lstinputlisting{code/process_get_available_pid.cpp}
    \caption{Code excerpt showing \code{process\_get\_available\_pid}.}
    \label{code:process_get_available_pid}
\end{figure}
Should it happen that our process table is at its maximum we had some
difficulty deciding what to do. As it stands we simply return -1, which is not
a valid array index, and could cause the system to fail. Among the
alternatives discussed in the group were kernel panic or waiting for a process
id to become available.

% TODO: TEST IF MORE THAN 32 PROCESSES EXISTS, WHAT HAPPENS? IS syscall_exec
%       SIMPLY GIVING BACK AN ERROR-CODE, OR DOES THE ENTIRE SYSTEM CRASH ?

\subsubsection{Starting processes}
Most of the implementation for starting processes was already in place. We
did, however, need to make some slight adjustments in order to accommodate
userland processes.

The original implementation of \code{process\_start} took a \code{const char
*} as its argument, which was used to pass the executable filename. We
modified the function to take a process id instead. Doing so, we recognize the
need of associating a process with an executable, hence reasoning the
declaration of such in the process control block structure
\figref{code:process_control_block_t}. In the code excerpt one can see that we
now reference the executable associated with that process on line 29. The
thread table is updated to make the process run on lines 15--16.

\begin{figure}[H]
    \lstinputlisting{code/process_start_p0.cpp}
    \vspace{-0.25in}\center{\dots}
    \lstinputlisting[firstnumber=29]{code/process_start_p1.cpp}
    \vspace{-0.25in}\center{\dots}
    \lstinputlisting[firstnumber=120]{code/process_start_p2.cpp}
    \caption{Code excerpt showing \code{process\_start}.}
    \label{code:process_start}
\end{figure}
At the very end of the \code{process\_start} function we have verified
numerous thing (most of which was already implemented in the original), which
allows us to run the process. Hence, we set its state as such on line 120.

\subsubsection{Finishing processes}
When a process returns, it calls \code{syscall\_exit} implicitly, which is
handled by \code{handle\_syscall\_exit} \figref{code:handle_syscall_exit},
which turn calls \code{process\_finish}.

Inside this function we would like to ensure that killing this process doesn't
affect other processes that may be related to it somehow. One relation is that
of sub-processes, or children. So, on lines 8--15 we look through the process
table, and if any process in the table has a parent reference to the process
we are about to kill, we remove it by setting its \code{parent\_id} to -1.

We then set the return-value of the process on line 20, as supplied via the
function argument from \code{handle\_syscall\_exit} and set its state to
zombie on line 21, which allows the parent process to kill it
\figref{code:process_join}.

\begin{figure}[H]
    \lstinputlisting{code/process_finish.cpp}
    \caption{Code excerpt showing \code{process\_finish}.}
    \label{code:process_finish}
\end{figure}

Note that some preparation is made for virtual memory on lines 27--28, which
must be done before finally calling \code{thread\_finish} on line 32, which
kills the thread on which the process was running.

\subsubsection{Joining processes}
In joining processes the current process must sit and wait for a child process
to finish up, the process id of which has been supplied as an argument to the
function via \code{handle\_syscall\_join} \figref{code:handle_syscall_join}.

The first thing we do is check whether or not the child process is alive or
not on lines 5--6. If not, we return an error, otherwise we proceed.

We put the current process in a waiting state and enter a loop in which we
put the process to sleep and continue running on another thread. We do this
until the child process becomes a zombie, which signals that it has finished
its work \figref{code:process_finish}. Once out of this loop we can save the
child process return-value, set the child process state to being dead and
reset the current process state to being running.

\begin{figure}[H]
    \lstinputlisting{code/process_join.cpp}
    \caption{Code excerpt showing \code{process\_join}.}
    \label{code:process_join}
\end{figure}
