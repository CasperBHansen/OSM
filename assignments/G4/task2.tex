%
% task2.tex
%
% Dynamic allocation for user processes.
%

\section{Dynamic Memory}
In order to support dynamic memory for user processes, we must make a few
adjustments to our current implementation. We must add a pointer to the end of
the heap to the process control block, the extension of this structure is
shown below.

\codefig{revised-pcb}{proc/process.h}{64}{83}
{Code excerpt showing the revised \code{process\_control\_block\_t} structure.}

Note that \code{heap\_ptr\_t} is declared as \code{void} on line 43 of the
same file.

After a few debug print outs we found that the stack pointer grows downward,
and thus starts out in high address space. Therefore, we would like our heap
to grow in the opposite direction, and from low address space. We found that
the lowest possible address from which we can initialize our heap pointer from
is immediately following the static data written from the executable into
memory on lines 125--137 in \file{proc/process.c}. So, we calculated the
address following that those segments, and initialized our heap pointer to
that address.

\codefig{heap_end-init}{proc/process.c}{194}{194}
{Code excerpt showing the initialization of \code{heap\_end}.}

\subsection{Implementation of \code{syscall\_memlimit}}
We extract the argument out of CPU register \code{A1} on line 81, and get the
current heap pointer from the process on line 82. The assignment required that
if the argument given is \code{NULL} then we simply return the current heap
pointer. This is done on lines 84--87. Also, the assignment specifically says
that we should not handle decreasing the heap, since \code{vm\_unmap} hasn't
been implemented yet, and that functionality would rely on its implementation.
Therefore, on line 90 we issue a kernel panic if it is the case that
\code{syscall\_memlimit} should actually decrease the heap.

Now that the preliminary checks have been done, we can proceed to allocate the
necessary pages. We do this by first calculating how many pages are needed to
fulfill the request on line 94. We then increase the current heap pointer by
the required amount on line 101. And then on lines 103--109, as the comment
suggests, we have shamelessly stolen much of it from how we saw virtual
memory was allocated in \file{proc/process.c}, and adapted it to suit our
needs. It should allocate at least the amount of pages needed to fulfill the
requested memory, and from our tests it seems to do so.

Lastly, we update the heap pointer for the process and return its virtual
memory address on lines 113--114.

\codefig{heap_end-init}{proc/syscall.c}{80}{115}
{Code excerpt showing the \code{syscall\_memlimit} implementation.}

\subsection{Implementation of \code{malloc}}
Most of the fixed heap size implementation could be reused. Only slight
adjustments were made in order to use the actual heap. First and foremost, we
commented out any references to the fixed size heap, which obviously gave us a
lot of compiler errors, but then we knew which parts of the algorithms were
dependent on it. Namely, the \code{heap\_init} function, which set the free
list values. These were altered to reflect that we were now referencing the
actual heap and the size would initially be zero.

\codefig{heap_init}{tests/lib.c}{743}{748}
{Code excerpt showing the revised \code{heap\_init} function.}

Now we will have to maintain the size and next pointer whenever we change the
heap. For already allocated memory this would still be handled by the loop in
\code{malloc} which looks for available free blocks. If no free blocks exist
we must allocate it and update the data accordingly. When exiting the
mentioned loop, the pointer \code{block} points to the last element in the
list; this is the pointer we wish to allocate memory for and return the
address of the data, which in this implementation using the block structure
would be stored at an offset of \code{size\_t} from the \code{block} address.

\codefig{malloc}{tests/lib.c}{791}{799}
{Code excerpt showing the extension of \code{malloc}.}

\subsection{Tests}
Our tests of \code{malloc} show a faulty implementation, since the results
have been unreliable. For simple allocations it works fine, but as the output
log \file{logs/mem\_test.log} shows, some data entries into the \code{ints}
array of the \file{tests/mem\_list} program are garbage values. We did not
find the erroneous code that causes this.

